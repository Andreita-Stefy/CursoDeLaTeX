%clase de texto matemático
% Clase del documento
\documentclass[a5paper, 10pt]{article}

% Paquetes
\usepackage[utf8]{inputenc}
\usepackage[spanish]{babel}
\usepackage[top=2cm, left=2cm]{geometry}
\usepackage{amsmath, amssymb, amsfonts, latexsym}
\usepackage{graphicx}
\usepackage{color}
\usepackage{multicol}
\usepackage{nicefrac}

% Comandos
\parindent = 0mm

\author{Andrea Chumaña}

\title{Estrucutra de texto matemático}

\date{\today}

% Contenido

\begin{document}
\maketitle
La optimización de funciones no es un tema analizado únicamente con herramientas del cálculo en una variable y de la programación lineal. Esta se puede generalizar a espacios más generales como son los espacios de Banach. A continuación se presenta el siguiente problema de optimización:
\begin{align}
\text{mín J(u,y,a)} =
\int_{0}^{\text {a}} \left(u^{'}(x)\right)^2 \, \text{dx} + \int_{0}^{\text {a}} y(x)^2 \, \text{dx} +
\dfrac{\text{a}^2}{\text{med}(0, \text{a}, \text{a}^2)},
\end{align}
\begin{center}
sujeta a 
 \begin{equation*}
\left\lbrace
\begin{array}{ll}
\text{-u}^{''} + \alpha (x) \text{u}(x)= y(x) \quad \text{en (0, a)},\\
\text{u}= 0	\qquad  \text{en} \big\{0, \text{a}\big\},\\
	\displaystyle \lim_{x\to 0} y(x)= \text{a},\\
	\text{a} \geq 4. 
\end{array}
\right.
\end{equation*}
\end{center}

La idea es optimizar sobre el conjunto de funciones de cada intervalo de la forma \([0,\text{a}]\)y determinar en el valor de \(\text{a} \geq 4\) que indique el mejor intervalo de trabajo.
 
\end{document}