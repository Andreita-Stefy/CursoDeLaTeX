\documentclass[a5paper,doc,12pt,apacite]{apa6}


\usepackage[utf8]{inputenc}
\usepackage[spanish,es-nolayout,es-nodecimaldot,es-tabla]{babel}
\usepackage[T1]{fontenc}
\usepackage[T1]{fontenc}


% Información del título, institución y fecha
\author{Elisa Kiseljak}
\affiliation{Tres historias europeas}
\title{epílogo}
\shorttitle{}
\date{\today}
\geometry{top=2.5cm, bottom=2cm, left=2cm, right=2cm}
\pagenumbering{Roman}


% Contenido
\begin{document}
	\maketitle
	\input{Texto}
   
	\maketitle  % Tres historias europeas 
La vida ésta. Absurda recorre sin espejos el trazado de su propio laberinto. Y avanza, temerosa del cielo y de la tierra, de su propio reflejo sobre el velo, del agua que esconde los recuerdos.

La vida, ésta. Trata de reconocerse en los fragmentos desguazados de la luna, atrapados con cazamariposas gigantes que nadie es capaz de sostener en vuelo sin cubrirse los ojos de arena.

La vida, ésta. Atrofiada y a medias en el reparto de patines, de alas, de olfatos, de espadas, madrigueras. Asustada camina caminos en silencio. Olvidando las estatuas de sal en las esquinas, las mujeres muertas de los árboles, los pozos secos de su voz aguda. Y tiembla.

La vida, ésta. Se detiene a descansar junto a las puertas y olvida innecesarias contraseñas, imprescindibles para hacerse pequeña y atravesar paredes desconchadas de goteras de alientos abatidos.

La vida, ésta. Sonora y agotada se derrumba en fosas de ruinas dibujadas con paletas de colores de madera. Y busca una estrella muerta en la que esconder el corazón antes de irse. Suspicaz ante el atraso y la memoria. Gélida ante lamentos seductores. Chiquita ante pieles expuestas en los rincones de las orillas de los mares que recuerda. Nostálgica sucumbe ante su propio miedo. La vida, ésta.

Sentada, reposa y espera, con sus ojos de puñal clavados en la arena, ansiosa, atenta. Buscando cualquier nube que la quiera, que la esconda, la proteja. Y suspira, cansada y ya dormida, frente a imágenes que empieza a hacer de piedra. Soñando un beso, un solo beso, una gruta entera de belleza. 




% Información de bibliografía y citas

\nocite{Bosch}
	% Encoding: UTF-8
	@Book{Bosch,
 	 title     = {Tres historias europeas},
  	publisher = {Caballo de Troya},
  	year      = {2005},
  	author    = {Lolita Bosch},
 	 address   = {Madrid},
  	isbn      = {84-934195-4-0},
	}

	@Comment{jabref-meta: databaseType:bibtex;}
	
	
\bibliographystyle{apacite} 
\bibliography{Libro}
\maketitle
\end{document}





