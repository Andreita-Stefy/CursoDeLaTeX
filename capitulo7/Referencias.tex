\documentclass[11pt]{article}

\usepackage[utf8]{inputenc}
\usepackage[T1]{fontenc}
\usepackage[spanish]{babel}
\usepackage{amsmath,amssymb,amsthm}

\usepackage{makeidx} % Este es el paquete a utilizarse

\usepackage[a4paper,margin=2.5cm]{geometry}
\usepackage[bookmarks=true,colorlinks=true,urlcolor=blue,linkcolor=blue,pdfstartview=FitB]{hyperref}

%
\renewcommand{\qedsymbol}{$\blacksquare$}
%
\theoremstyle{plain}%
\newtheorem{thm}{Teorema}[section]
\newtheorem{prop}[thm]{Proposición}
%
\theoremstyle{definition}%
\newtheorem{defn}{Definición}[section]
%

\makeindex

\title{Referencias e índices}
\author{Milton Torres}
\date{\today}

\begin{document}
\maketitle
\tableofcontents

\section{Topología}
\begin{defn}[Espacio topológico]\index{Espacio!topológico}
	Una familia \(\mathcal{T}\) de partes de un conjunto \(X\) define una topología sobre \(X\) si las tres condiciones son verificadas:
	\begin{enumerate}
		\item El conjunto vacío \(\emptyset\) y el conjunto \(X\) pertenecen a \(\mathcal{T}\);
		\item la intersección finita de elementos de \(\mathcal{T}\) pertenece a \(\mathcal{T}\); y
		\item la reunión cualquiera de elementos de \(\mathcal{T}\) pertenece a \(\mathcal{T}\).
	\end{enumerate}
	Los elementos de \(\mathcal{T}\) son llamados \emph{conjuntos abiertos}, sus complementarios, \emph{cerrados}; y el espacio \((X, \mathcal{T})\), \emph{espacio topológico}.
	\index{Conjunto!cerrado} \index{Conjunto!abierto}
\end{defn}
Esto se lo pude encontrar en \cite{VK1997}, o más precisamente en \cite[p.\,3]{VK1997}.


\begin{defn}[Espacio métrico]\index{Espacio!métrico}
	Un espacio métrico \((E, d_{E})\) está dado por un conjunto \(E\) dotado de una aplicación  \(d_E : E \times E \longrightarrow [0,+\infty[\), llamada \emph{distancia} o \emph{métrica}, que verifica las siguientes propiedades:
	\begin{enumerate}
		\item[D.1] \textbf{Simetría}: para todo \(x, y\in E\), \(d_{E}(x,y) = d_{E} (y,x)\).
		\item[D.2] \textbf{Separabilidad}: \(d_{E}(x,y)=0\) si y solo si \(x=y\).
		\item[D.3] \textbf{Desigualdad triangular}: para todo \(x, y, z\in E\),
		\[
			d_{E}(x,y)\leq d_{E}(x,z)+d_{E}(z,y).
      		\]
	\end{enumerate}
	A los elementos de un espacio métrico se los denomina \emph{puntos}.
\end{defn}


\newpage
\section{Continuidad y otros}
\begin{prop}\index{Aplicación!continua}
	Una aplicación \(f\) de \((E, d_E)\) en \((F,d_F)\) es continua si y solo sí para toda sucesión convergente \((x_n)_{n\in\mathbb{N}}\) de los elementos de \(E\) se tiene
	\[
		f \left(\lim_{n \to +\infty} x_n \right) = \lim_{n \to +\infty} f(x_n)
	\]
\end{prop}
La verificación de este se lo apreciar en \cite{EK1989}. %Para referancia inteligente, me lleva directo a la refencia que estoy citando

\begin{thm}[Teorema fundamental del cálculo]\index{Teorema!fundamental del cálculo}
	Si \(f:[a,b]\longrightarrow \mathbb{R}\) es continua y si \(A(x)=\int_a^x\,f(t)\,dt\), entonces \(A'=f\). Es decir,
	\[
		A'_{+}(a)=f(a),\quad A'_{-}(b)=f(b)
	\]
	y para todo \(x \in ]a,b[\)
	\[
		A'(x)=f(x)
	\]
\end{thm}
\begin{proof}
 Esta demostración se la pude encontrar en \cite[pp.\,42--43]{GR2010}. %para citar con numero de pagina
\end{proof}



\nocite{TA2008,greenwade93} 
\bibliographystyle{apalike} % plain, abbrv, alpha, apa
\bibliography{Bibliografia}


\printindex


\newpage


\end{document} 