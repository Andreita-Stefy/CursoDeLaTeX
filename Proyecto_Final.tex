% Clase del documento
\documentclass[a4paper, 10pt]{article}

% Paquetes
\usepackage[utf8]{inputenc}
\usepackage[spanish, es-nolayout, es-nodecimaldot, es-tabla]{babel}
\usepackage[top=2cm, left=2cm, right=5cm]{geometry}
\usepackage{amsmath, amssymb, amsfonts, latexsym}
\usepackage{graphicx}
\usepackage{color}
\usepackage{multicol}
\usepackage{nicefrac}
\usepackage[x11names,table]{xcolor}
\usepackage{longtable}
\usepackage{booktabs}
\usepackage{url}


% Comandos
\parindent = 00mm
\title{ESCUELA POLITÉCNICA NACIONAL}
\author{FACULTAD DE CIENCIAS}
\date{\today}
\begin{document}
\maketitle
\begin{itemize}
 \item Nombre: Andrea Chumaña
       \begin{itemize}
        \item Proyecto Final de Latex.
         \item Ejercicio de Complementos de Cálculo.
     \end{itemize}
\end{itemize}
\section{Ejercicio}

{\color{magenta} Suponga que exite un índice \(n_0\in \mathbb{N}\) y  un número real \( K\) mayor que 0 tal que
\[ X_n > K
\]
\(\forall n > n_0\), que existe  un índice \(n \geq n_0 \), existe un índice \(n_1\) tal que
\[Y_n < 0
\]
\(\forall n > n_0\), y que la sucesión de término general \(y_n\) converge a 0.
Demuestre que la sucesión de término general
\[\dfrac{X_n}{Y_n} \rightarrow -\infty\]}

\subsection*{Demostración}
Supongamos que
\begin{align*}
\text{Existe}\quad n_0 \in \mathbb{N} \quad \text{y que} 
\quad K > 0 \quad \text{tal que}&
\end{align*}
\begin{equation}
X_n\geq K \qquad \forall n \geq n_0,
\end{equation}
\begin{equation*}
\text{Existe} \quad n_1 \in \mathbb{N} \quad \text{tal que}\quad \forall n \geq n_1
\end{equation*}
\begin{equation}
Yn < 0
\end{equation}
\\
\begin{equation}
Yn\rightarrow 0
\end{equation}

Para esto supongamos que 
\begin{equation}
M < 0 \\
\end{equation}

Vamos a demostrar que existe \(n \in \mathbb{N}\) tal que \( \forall n \geq n_2\)\\
\begin{equation*}
\dfrac{X_n}{Y_n} < M
\end{equation*}
De (1) junto con (4) tenemos 
\begin{equation}
\dfrac{-K}{M} > 0 ;
\end{equation}

Así, de (3), existe \(n_3 \in \mathbb{N}\) tal que \(\forall n \geq n_3 \)\\
\begin{equation}
|y_n - 0 | < \dfrac{-K}{M}
\end{equation}
Sí
\( n_4 = \text{máx} \lbrace n_1,n_3 \rbrace \)
entonces, \(\forall n\geq n_4 ;\)
\begin{equation*}
Y_n < 0
\end{equation*}
y
\begin{equation*}
|y_n - 0 | = |y_n| = - y_n < \dfrac{-K}{M};
\end{equation*}
es decir\\
\begin{equation}
\dfrac{-1}{Y_n}> \dfrac{-M}{K} \qquad \forall n\geq  n_4
\end{equation}
Finalmente, sì
\begin{center}
\(n_2 = \text{máx} \lbrace n_0,n_4 \rbrace ,\)
\end{center}
tenemos que \(\forall n\geq n_2\), se verifica (7)y (1)tenemos
\begin{equation*}
\dfrac{-X_n}{Y_n}> \dfrac{-M \cdot K}{K};
\end{equation*}
es decir 
\begin{equation*}
\dfrac{X_n}{Y_n} < M \qquad \forall n \geq n_2
\end{equation*}
\begin{flushright}
\textsc{como queríamos demostrar}
\end{flushright}
\end{document}
