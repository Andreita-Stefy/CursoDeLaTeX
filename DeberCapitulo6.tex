\documentclass[a3paper, 12pt]{article}
\usepackage[utf8]{inputenc}
\usepackage[spanish]{babel}
\usepackage[top=2cm, left=2cm]{geometry}
\usepackage{amsmath, amssymb, amsfonts, latexsym}
\usepackage{graphicx}
\usepackage{xcolor}
\usepackage{multicol}
\usepackage{nicefrac}

\author{Andrea Chumaña}
\title{Deber Capitulo 6}
\date{\today}

\begin{document}
\maketitle
La solución de la ecuación diferencial
\begin{center}
  \begin{equation*}
\text{y}^{'} + 2\text{y}(\text{x})= \left\lbrace
\begin{array}{ll}
\text{1} & \text{si} \quad \text{x} \in [0,3],\\
\text{0} & \text{si} \quad \text{x} > 3.
\end{array}\\
\right.
\end{equation*}
\text{Sujeto a} 
\center{\text{y} = 0}
\end{center}
{\color{magenta}Está dada} por la función a trozos 
\begin{center}
 \begin{equation*}
\text{y}(\text{x})= \left\lbrace
\begin{array}{ll}
\dfrac{1}{2}(\exp^{-2\text{x}}-1)& \text{si} \quad \text{x}\leq 3,\\
\\
\dfrac{1}{2}(\exp^{-2\text{x}}-\exp^{6-2\text{x})}& \text{si}  \quad \text{x} > 3.
\end{array}
\right.
\end{equation*}
\end{center}
\begin{figure}[h!] 
\centering 
\includegraphics [scale = 0.5]{grafico1}
\caption{ \emph{<<Funciónes a trozos>>}.}
\label{mec} 
\end{figure}
\end{document}